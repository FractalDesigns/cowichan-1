\section{A Critique of the C Implementation\label{s:critique}}

This section is a critique of the ANSI~C reference implementation of the problem suite.
It is intended to serve as a model for (self-)criticism of other programming systems.

\begin{itemize}
\item	C's support for multi-dimensional arrays (MDAs) is very weak.
	There is no way to dynamically allocate an MDA in a single go---one must either
	allocate a block of the same size as the desired array,
	and then do indexing calculations by hand,
	or allocate a vector of pointers to allocated vectors of pointers to{\ldots}to vectors holding data.
	MDAs do not carry dimension information with them,
	so it is impossible to determine the size of an array parameter within a function.
	Finally,
	C does not treat all axes of an array equally:
	while it is trivial to take a slice out of a 2-dimensional array along the most-significant axis,
	it is impossible to slice it along the other axis.
\item	C does not distinguish between Boolean and integer types.
	As a result,
	the {\tt{matrix}} and {\tt{mask}} arguments to the invasion percolation problem
	can be passed in reverse order without a type error.
	Using {\tt{typedef}} to create a Boolean type name does not solve this
	(at least, not in {\tt{gcc}} V2.5.8).
\item	Union types cannot be safely initialized.
	This complicated the implementation of the graphics interface,
	where it would have been much more elegant to define a union type,
	each of whose variants held parameter specifications for a single toy.
	The code in the graphics module {\tt{gfx.c}} relies instead on arrays of {\tt{int}}s and {\tt{float}}s,
	initializing some and filling other with don't-care values.
	This is neither safe nor elegant.
\item	Parameter values cannot be used in the initialization of local variables inside functions.
	In particular,
	it is not possible to create a local vector with a length specified by an input parameter.
	Such a facility would be useful in the {\tt{winnow}} toy,
	where we have instead allocated local temporaries of the maximum possible size.
\item	The automatic conversion of floats to doubles across function calls can quite often be a nuisance.
	For example, the ``fail()'' error-handling routine was buggy because
	all \verb`real` arguments were being automatically converted to doubles,
	but were being taken off the stack as floats.
\item	no intrinsic notion of group ID/self ID in threads
\item	packaging parameters for threads
\item	pointers to array vs.\ arrays themselves (semantics of definition depends upon context)
\end{itemize}
