\subsection{{\tt{invperc}}:
	Invasion Percolation
	\label{s:toys-invperc}}

Invasion percolation models the displacement of one fluid
(such as oil)
by another
(such as water)
in fractured rock \cite{b:percolation-theory}.
In two dimensions,
this can be simulated by generating
an $N{\times}N$ grid of random numbers in the range $[1{\ldots}R]$,
and then marking the center cell of the grid as filled.
In each iteration,
one examines the four orthogonal neighbors of all filled cells,
chooses the one with the lowest value
(i.e., the one with the least resistance to filling),
and fills it in.
Figure~\ref{f:invperc1} shows the first few steps in this process,
while Figure~\ref{f:invperc2} shows the evolution of the final fractal shape.

\begin{figure}
\begin{footnotesize}
\begin{math}
\begin{array}{ccc}
\begin{array}{|c|c|c|c|c|}
\hline
26 & 12 & 72 & 45 & 38 \\
\hline
10 & 38 & 39 & 92 & 38 \\
\hline
44 & 29 & \circ & 29 & 77 \\
\hline
61 & 26 & 90 & 35 & 11 \\
\hline
83 & 84 & 18 & 56 & 52 \\
\hline
\end{array}
&
\begin{array}{|c|c|c|c|c|}
\hline
26 & 12 & 72 & 45 & 38 \\
\hline
10 & 38 & 39 & 92 & 38 \\
\hline
44 & \star & \circ & 29 & 77 \\
\hline
61 & 26 & 90 & 35 & 11 \\
\hline
83 & 84 & 18 & 56 & 52 \\
\hline
\end{array}
&
\begin{array}{|c|c|c|c|c|}
\hline
26 & 12 & 72 & 45 & 38 \\
\hline
10 & 38 & 39 & 92 & 38 \\
\hline
44 & \circ & \circ & 29 & 77 \\
\hline
61 & \star & 90 & 35 & 11 \\
\hline
83 & 84 & 18 & 56 & 52 \\
\hline
\end{array}
\\
 & & \\
\begin{array}{|c|c|c|c|c|}
\hline
26 & 12 & 72 & 45 & 38 \\
\hline
10 & 38 & 39 & 92 & 38 \\
\hline
44 & \circ & \circ & \star & 77 \\
\hline
61 & \circ & 90 & 35 & 11 \\
\hline
83 & 84 & 18 & 56 & 52 \\
\hline
\end{array}
&
\begin{array}{|c|c|c|c|c|}
\hline
26 & 12 & 72 & 45 & 38 \\
\hline
10 & 38 & 39 & 92 & 38 \\
\hline
44 & \circ & \circ & \circ & 77 \\
\hline
61 & \circ & 90 & \star & 11 \\
\hline
83 & 84 & 18 & 56 & 52 \\
\hline
\end{array}
&
\begin{array}{|c|c|c|c|c|}
\hline
26 & 12 & 72 & 45 & 38 \\
\hline
10 & 38 & 39 & 92 & 38 \\
\hline
44 & \circ & \circ & \circ & 77 \\
\hline
61 & \circ & 90 & \circ & \star \\
\hline
83 & 84 & 18 & 56 & 52 \\
\hline
\end{array}
\\
 & & \\
\begin{array}{|c|c|c|c|c|}
\hline
26 & 12 & 72 & 45 & 38 \\
\hline
10 & \star & 39 & 92 & 38 \\
\hline
44 & \circ & \circ & \circ & 77 \\
\hline
61 & \circ & 90 & 35 & 11 \\
\hline
83 & 84 & 18 & 56 & 52 \\
\hline
\end{array}
&
\begin{array}{|c|c|c|c|c|}
\hline
26 & 12 & 72 & 45 & 38 \\
\hline
\star & \circ & 39 & 92 & 38 \\
\hline
44 & \circ & \circ & \circ & 77 \\
\hline
61 & \circ & 90 & \circ & 11 \\
\hline
83 & 84 & 18 & 56 & 52 \\
\hline
\end{array}
&
\begin{array}{|c|c|c|c|c|}
\hline
26 & \star & 72 & 45 & 38 \\
\hline
\circ & \circ & 39 & 92 & 38 \\
\hline
44 & \circ & \circ & \circ & 77 \\
\hline
61 & \circ & 90 & \circ & \circ \\
\hline
83 & 84 & 18 & 56 & 52 \\
\hline
\end{array}
\end{array}
\end{math}
\end{footnotesize}
\caption{Invasion Percolation\label{f:invperc1}}
\end{figure}

\begin{figure}
\begin{center}
% {\epsfxsize=0.2\textwidth\framebox{\epsffile{fig/invperc1.ps}}}
% {\epsfxsize=0.2\textwidth\framebox{\epsffile{fig/invperc2.ps}}}
% {\epsfxsize=0.2\textwidth\framebox{\epsffile{fig/invperc3.ps}}}
% {\epsfxsize=0.2\textwidth\framebox{\epsffile{fig/invperc4.ps}}}
\end{center}
\caption{Fractal Generated by Invasion Percolation\label{f:invperc2}}
\end{figure}

The simulation continues until
some fixed percentage of cells have been filled,
or until some other condition
(such as the presence of trapped regions)
is achieved.
The fractal structure of the filled and unfilled regions
is then examined to determine how much oil could be recovered.
The na\"{\i}ve way to implement this is to repeatedly scan the array;
a more sophisticated, and much faster, sequential technique
is to maintain a priority queue of unfilled cells
which are neighbors of filled cells.
This latter technique is similar to the list-based methods used in some cellular automaton programs,
and is very difficult to parallelize effectively.

The inputs to this module are:
\begin{description}
\item[{\tt{matrix}}:]
	an integer matrix.
\item[{\tt{nrows, ncols}}:]
	the number of rows and columns in the matrix and mask.
\item[{\tt{nfill}}:]
	the number of points to fill.
\end{description}
Its output is:
\begin{description}
\item[{\tt{mask}}:]
	a Boolean matrix filled with {\tt{true}} (showing a filled cell) or {\tt{false}} (showing an unfilled cell).
\end{description}
Filling begins at the central cell of the matrix
(rounding down for even-sized axes).
