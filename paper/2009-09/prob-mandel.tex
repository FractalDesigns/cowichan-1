\subsection{{\tt{mandel}}:
	Mandelbrot Set Generation
	\label{s:toys-mandel}}

\begin{figure}
% \begin{center}\framebox{\epsffile{fig/mandel.ps}}\end{center}
\caption{The Mandelbrot Set\label{f:mandel}}
\end{figure}

This module generates the Mandelbrot Set
for a specified region of the complex plane
(Figure~\ref{f:mandel}).
The inputs to this module are:
\begin{description}
\item[{\tt{nrows, ncols}}:]
	the number of rows and columns in the matrix.
\item[{\tt{x0, y0}}:]
	the real coordinates of the lower-left corner
	of the region to be generated.
\item[{\tt{dx, dy}}:]
	the extent of the region to be generated.
\end{description}
Its output is:
\begin{description}
\item[{\tt{matrix}}:]
	an integer matrix containing the iteration count at each point in the region.
\end{description}

Given initial coordinates $(x_0, y_0)$,
the Mandelbrot Set is generated by iterating the equation
\begin{eqnarray*}
x^{\prime}	& =	& x^2 - y^2 + y_0	\\
y^{\prime}	& =	& 2{x}{y} + x_0
\end{eqnarray*}
until either an iteration limit is reached,
or the values diverge.
The iteration limit used in this module is 150 steps;
divergence occurs when $x^2 + y^2$ becomes 2.0 or greater.
The integer value of each element of the matrix is
the number of iterations done.
Note that,
as in all problems,
the output is required to be
independent of the number of processors used.
This requires programmers to be careful about
floating-point roundoff effects.
