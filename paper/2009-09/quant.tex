\section{Quantifying the Importance of Usability\label{s:quant}}

According to Amdahl's Law,
if $t(n)$ is an algorithm's running time on $n$ processors
and $\sigma$ is the algorithm's serial fraction
(i.e., the fraction of the algorithm that cannot be parallelized),
then:
\begin{eqnarray*}
t(p) & = & {\sigma}t(1) + (1 - {\sigma})t(1)/p
\end{eqnarray*}
The speedup on $p$ processors is therefore:
\begin{eqnarray*}
s(p) & = & t(1)/t(p) \\
     & = & t(1)/({\sigma}t(1) + (1 - {\sigma})t(1)/p) \\
     & = & 1/({\sigma} + (1 - {\sigma})/p)
\end{eqnarray*}
As $p{\leftarrow}{\infty}$, the speedup is bounded by $1/{\sigma}$.
This makes intuitive sense:
if 10\% of an algorithm can't be parallelized,
then even with infinite hardware,
the program can only be made to run 10 times faster.

But software doesn't just happen:
it has to be written and debugged,
and these activities usually aren't made faster by parallel hardware\footnote{In fact,
most programmers believe that doing parallel programming is harder and slower than doing it sequentially.}.
If we let $T$ and $S$ be the equivalents of $t$ and $s$ over the program's whole lifetime,
and $D$ be its total development time,
then:
\begin{eqnarray*}
T(p) & = & {\sigma}T(1) + (1 - {\sigma})t(1)/p + D
\end{eqnarray*}
The achievable \emph{lifetime} speedup is then:
\begin{eqnarray*}
S(\infty) & = & 1/({\sigma} + D/T(1))
\end{eqnarray*}

The ratio of development time to (sequential) running time is therefore as important to speedup
as the serial fraction of the algorithm.
Put another way,
the time required to parallelize a program effectively increases the program's effective serial fraction.
Unless parallelization time can be substantially reduced,
parallel programming will only be cost-effective when:
\begin{itemize}
\item	the application is trivially parallel (e.g., task-farming a rendering calculation);
\item	the expected total program runtime is very large
	(i.e., the program is a package used extensively in a field
	such as computational chemistry or vehicle crash simulation);
	or
\item	cost is not an obstacle (i.e., the customer is Disney or the NSA).
\end{itemize}

The aim of this problem suite is to capture enough information about $D$
to allow competing parallel programming systems to be ranked and compared.

\subsection{Previous Work\label{s:previous}}

Two comparative efforts in parallel programming are \cite{b:babb-cases} and \cite{b:salishan}.
The first presented implementations of a simple numerical quadrature program
in more than a dozen different parallel languages used on mid-1980s hardware.
The second presented implementations of
the Salishan Problems---Hamming number generation,
isomer enumeration,
skyline matrix reduction,
and a simple discrete event simulator---in
C*~\cite{b:dataparallel-c}, Occam~\cite{b:occam}, Ada~\cite{b:ada},
and several dataflow and functional languages.
Both books conveyed the flavor of the languages they describe,
but neither compared languages or problem implementations directly.

More recently,
Lutz has explored quantitative comparison of the relative productivity
of different programming languages \cite{b:prechelt-languages,b:prechelt-platforms}.
Our long-term aim is to replicate his work in the domain of parallel programming.
