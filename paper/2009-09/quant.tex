\section{Quantifying the Importance of Usability\label{s:quant}}

According to Amdahl's Law,
the time $t$ required to execute a program has two components:
$t_{seq}$, representing sequential operations that cannot be parallelized,
and $t_{par}$, representing those operations which can.
If we let $\sigma$ be the ``serial fraction'' of the program,
so that $t_{seq} = {\sigma}t$ and $t_{par} = (1-{\sigma})t$,
then the speedup which can be achieved using $P$ processors is:
\begin{eqnarray*}
s(P)	& =	& \frac{t(1)}{t(P)}	\\
	& =	& \frac{t}{{{\sigma}t} + {{(1-{\sigma})t}}/{P}}
\end{eqnarray*}
As $P{\rightarrow}{\infty}$,
the potential speedup is bounded by $1/{\sigma}$.

What is not usually acknowledged is
the influence of programming system usability on
the potential speedup of sequential programs.
As stated above, Amdahl's Law applies only to a single run of a program.
In the real world, programs are first developed, and then run many times.
If we let $T$ be the total running time of a program over its lifetime,
$D_{seq}$ be the time required to develop the sequential components of a program,
and $D_{par}$ be the time required to develop the parallel components
(or parallelize the sequential components),
then the lifetime speedup we can achieve is given by:
\begin{eqnarray*}
S(P)	& =	& \frac{D_{seq} + T(1)}{D_{seq} + D_{par} + T(P)}	\\
	& =	& \frac{D_{seq} + T}{D_{seq} + D_{par} + {{\sigma}T} + {{(1-{\sigma})T}}/{P}}
\end{eqnarray*}
Assume that we are parallelizing a ``dusty deck'',
i.e.\ that the time $D_{seq}$ has already been invested.
If we let $\phi = D_{par}/T$ be the ratio of parallelization time to total program runtime,
then achievable speedup is limited by:
\begin{eqnarray*}
S_{legacy}(P)	& =	& \frac{T}{D_{par} + {\sigma}T}		\\
		& =	& \frac{1}{{\phi} + {\sigma}}
\end{eqnarray*}
The time required to parallelize a program can therefore be seen
as increasing that program's effective serial fraction.
Unless parallelization time can be substantially reduced,
massively-parallel computing is likely to remain attractive only in those cases in which:
\begin{itemize}
\item	the application is trivially parallel (e.g.\ task-farming a rendering calculation);
\item	the expected total program runtime is very large
	(i.e.\ the program is a package used extensively in a field
	such as computational chemistry or vehicle crash simulation);
	or
\item	cost is not an obstacle (i.e.\ the customer is the NSA).
\end{itemize}

\subsection{Previous Work\label{s:previous}}

In a 1981 survey article \cite{b:psych-prog-survey}, B.~A.\ Sheil observed that:
\begin{quote}
As practiced by computer science,
the study of programming is an unholy mixture of mathematics,
literary criticism,
and folklore.
\end{quote}
Little has changed since.
Computer scientists pay lip service to the importance of usability,
but are strangely reluctant to try to quantify it,
particularly with respect to the tools they use themselves.
Religious debates about the ``right'' first-year teaching language
or the ``best'' GUI toolkit are common;
systematic use of the experimental method is not.

Most of the experiments which have been done have concentrated on ``programming in the small'' issues,
such as readability and code-pattern recognition
\cite{b:reverse-parsing-2,b:cloze-program-1,b:prog-comprehension-1}.
The little work that has been done on ``programming in the large'' has been anecdotal,
such as \cite{b:compare-ada-c-pascal}.
This book presented various authors' impressions of Pascal, C, and Ada
(before any complete Ada implementations existed).
Its editors formulated a list of questions to use in comparing languages,
but did not use it themselves.
A modified version of this questionnaire,
tailored for parallel programming languages,
is given in Appendix~\ref{s:questions},
and will be used as a guideline in our work.

Two comparative efforts in parallel programming are \cite{b:babb-cases} and \cite{b:salishan}.
The first presented implementations of a simple numerical quadrature program
in more than a dozen different parallel languages
in use on mid-1980s hardware.
The second presented implementations of
the Salishan Problems---Hamming number generation,
isomer enumeration,
skyline matrix reduction,
and a simple discrete event simulator---in
C* \cite{b:dataparallel-c}, Occam~\cite{b:occam}, Ada~\cite{b:ada},
and a variety of dataflow and functional languages.
Both of these books convey the flavor of the languages they describe,
but neither made any effort to compare languages or problem implementations directly.

Inspired by the Salishan Problems,
the first author developed a suite of seven problems \cite{b:cowichan-1}
in order to test the limits of parallel programming systems.
Individual problems were badly load-balanced,
manipulated irregular pointer-based structures,
and so on.
Our experience implementing them was extremely valuable,
as it uncovered many bugs in the programming system used,
and revealed several shortcomings of the system's design.
However,
we also discovered that implementing each problem takes six to eight weeks,
which we feel is prohibitive at this exploratory stage.
