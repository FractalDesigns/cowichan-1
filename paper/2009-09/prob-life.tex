\subsection{{\tt{life}}:
	Game of Life
	\label{s:toys-life}}

\begin{figure}
\begin{center}
% {\epsfxsize=0.2\textwidth\framebox{\epsffile{fig/life1.ps}}}
% {\epsfxsize=0.2\textwidth\framebox{\epsffile{fig/life2.ps}}}
% {\epsfxsize=0.2\textwidth\framebox{\epsffile{fig/life3.ps}}}
% {\epsfxsize=0.2\textwidth\framebox{\epsffile{fig/life4.ps}}}
\end{center}
\caption{Snapshots from the Game of Life\label{f:life}}
\end{figure}

This module simulates the evolution of Conway's Game of Life,
a two-dimensional cellular automaton (Figure~\ref{f:life}).
Its inputs are:
\begin{description}
\item[{\tt{matrix}}:]
	a Boolean matrix representing the Life world.
\item[{\tt{nrows, ncols}}:]
	the number of rows and columns in the matrix.
\item[{\tt{numgen}}:]
	the number of generations to simulate.
\end{description}
Its output is:
\begin{description}
\item[{\tt{matrix}}:]
	a Boolean matrix representing the world after simulation.
\end{description}
At each time step,
this module must count the number of live ({\tt{true}}) neighbors
of each cell,
using both orthogonal and diagonal connectivity
and circular boundary conditions.
The update rule is simple:
if a cell has 3 live neighbors,
or has 2 live neighbors and is already alive,
it is alive in the next generation.
In any other situation,
the cell becomes, or stays, dead.
