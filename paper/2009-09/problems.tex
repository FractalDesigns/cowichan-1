\section{The Cowichan Problems\label{s:toys}}

This section describes the problems making up the full problem suite.
As discussed above,
the implementation of each is to be used both as a stand-alone program
and as a module in the chained problem sequence described in Section~\ref{s:issues-chain}.
Table~\ref{t:toy-codes} lists the one-character codes sometimes used to identify toys.

\begin{table}
\begin{center}
\begin{tabular}{cl|cl}
Code	& Toy				& Code	& Toy				\\
\hline
C	& convex hull			& O	& outer product			\\
G	& Gaussian elimination		& P	& matrix-vector product		\\
H	& halving shuffle		& R	& random matrix generation	\\
I	& invasion percolation		& S	& successive over-relaxation	\\
L	& Game of Life			& T	& image thresholding		\\
M	& Mandelbrot Set		& V	& vector difference		\\
N	& point normalization		& W	& point value winnowing
\end{tabular}
\caption{Single-Character Codes for Toys\label{t:toy-codes}}
\end{center}
\end{table}

\subsection{{\tt{convex}}:
	Convex Hull
	\label{s:toys-convex}}

This module takes a list of two-dimensional points and reorders them by
doing multiple convex hull computations. The convex hull is the boundary of
the minimal convex set containing a given non-empty finite set of points
in the plane. In other words, all points not in the convex hull are
enclosed in the convex hull polygon. At each step the convex hull points
are taken out of the input list and are put into the output list. The
computation terminates when there are no more points left in the input
list.  Inputs are:

\begin{description}
\item[{\tt{pointsIn}}:]
	the vector of input points.
\end{description}
Its output is:
\begin{description}
\item[{\tt{pointsOut}}:]
	the vector of output points (a permutation of the input).
\end{description}

\subsection{{\tt{gauss}}: Gaussian Elimination\label{s:toys-gauss}}

This module solves a matrix equation $AX=V$ for
a dense, symmetric, diagonally dominant matrix $A$
and an arbitrary vector non-zero $V$
using explicit reduction.
Input matrices are required to be symmetric and diagonally dominant
in order to guarantee that there is a well-formed solution to the equation.

{\inputspec}

\begin{description}
\item[{\tt{matrix}}:]
	the real matrix $A$.
\item[{\tt{target}}:]
	the real vector $V$.
\end{description}

{\outputspec}

\begin{description}
\item[{\tt{solution}}:]
	a real vector containing the solution $X$.
\end{description}

\subsection{{\tt{half}}:
	Two-Dimensional Shuffle
	\label{s:toys-half}}

This module divides the values in a rectangular two-dimensional integer matrix into two halves along one axis,
shuffles them,
and then repeats this operation along the other axis.
Values in odd-numbered locations are collected at the low end of each row or column,
while values in even-numbered locations are moved to the high end.
An example transformation is:

\begin{eqnarray*}
\begin{array}{cccc}
a & b & c & d \\
e & f & g & h \\
i & j & k & l
\end{array}
& = &
\begin{array}{cccc}
a & c & b & d \\
i & k & j & l \\
e & g & f & h
\end{array}
\end{eqnarray*}

Note that how an array element is moved depends only on whether its location index is odd or even,
not on whether its value is odd or even.

The inputs to this module are:
\begin{description}
\item[{\tt{matrix}}:]
	an integer matrix.
\item[{\tt{nrows, ncols}}:]
	the number of rows and columns in the matrix.
\end{description}
Its output is:
\begin{description}
\item[{\tt{matrix}}:]
	an integer matrix containing shuffled values.
\end{description}

\subsection{{\tt{invperc}}:
	Invasion Percolation
	\label{s:toys-invperc}}

Invasion percolation models the displacement of one fluid
(such as oil)
by another
(such as water)
in fractured rock \cite{b:percolation-theory}.
In two dimensions,
this can be simulated by generating
an $N{\times}N$ grid of random numbers in the range $[1{\ldots}R]$,
and then marking the center cell of the grid as filled.
In each iteration,
one examines the four orthogonal neighbors of all filled cells,
chooses the one with the lowest value
(i.e., the one with the least resistance to filling),
and fills it in.
Figure~\ref{f:invperc1} shows the first few steps in this process,
while Figure~\ref{f:invperc2} shows the evolution of the final fractal shape.

\begin{figure}
\begin{footnotesize}
\begin{math}
\begin{array}{ccc}
\begin{array}{|c|c|c|c|c|}
\hline
26 & 12 & 72 & 45 & 38 \\
\hline
10 & 38 & 39 & 92 & 38 \\
\hline
44 & 29 & \circ & 29 & 77 \\
\hline
61 & 26 & 90 & 35 & 11 \\
\hline
83 & 84 & 18 & 56 & 52 \\
\hline
\end{array}
&
\begin{array}{|c|c|c|c|c|}
\hline
26 & 12 & 72 & 45 & 38 \\
\hline
10 & 38 & 39 & 92 & 38 \\
\hline
44 & \star & \circ & 29 & 77 \\
\hline
61 & 26 & 90 & 35 & 11 \\
\hline
83 & 84 & 18 & 56 & 52 \\
\hline
\end{array}
&
\begin{array}{|c|c|c|c|c|}
\hline
26 & 12 & 72 & 45 & 38 \\
\hline
10 & 38 & 39 & 92 & 38 \\
\hline
44 & \circ & \circ & 29 & 77 \\
\hline
61 & \star & 90 & 35 & 11 \\
\hline
83 & 84 & 18 & 56 & 52 \\
\hline
\end{array}
\\
 & & \\
\begin{array}{|c|c|c|c|c|}
\hline
26 & 12 & 72 & 45 & 38 \\
\hline
10 & 38 & 39 & 92 & 38 \\
\hline
44 & \circ & \circ & \star & 77 \\
\hline
61 & \circ & 90 & 35 & 11 \\
\hline
83 & 84 & 18 & 56 & 52 \\
\hline
\end{array}
&
\begin{array}{|c|c|c|c|c|}
\hline
26 & 12 & 72 & 45 & 38 \\
\hline
10 & 38 & 39 & 92 & 38 \\
\hline
44 & \circ & \circ & \circ & 77 \\
\hline
61 & \circ & 90 & \star & 11 \\
\hline
83 & 84 & 18 & 56 & 52 \\
\hline
\end{array}
&
\begin{array}{|c|c|c|c|c|}
\hline
26 & 12 & 72 & 45 & 38 \\
\hline
10 & 38 & 39 & 92 & 38 \\
\hline
44 & \circ & \circ & \circ & 77 \\
\hline
61 & \circ & 90 & \circ & \star \\
\hline
83 & 84 & 18 & 56 & 52 \\
\hline
\end{array}
\\
 & & \\
\begin{array}{|c|c|c|c|c|}
\hline
26 & 12 & 72 & 45 & 38 \\
\hline
10 & \star & 39 & 92 & 38 \\
\hline
44 & \circ & \circ & \circ & 77 \\
\hline
61 & \circ & 90 & 35 & 11 \\
\hline
83 & 84 & 18 & 56 & 52 \\
\hline
\end{array}
&
\begin{array}{|c|c|c|c|c|}
\hline
26 & 12 & 72 & 45 & 38 \\
\hline
\star & \circ & 39 & 92 & 38 \\
\hline
44 & \circ & \circ & \circ & 77 \\
\hline
61 & \circ & 90 & \circ & 11 \\
\hline
83 & 84 & 18 & 56 & 52 \\
\hline
\end{array}
&
\begin{array}{|c|c|c|c|c|}
\hline
26 & \star & 72 & 45 & 38 \\
\hline
\circ & \circ & 39 & 92 & 38 \\
\hline
44 & \circ & \circ & \circ & 77 \\
\hline
61 & \circ & 90 & \circ & \circ \\
\hline
83 & 84 & 18 & 56 & 52 \\
\hline
\end{array}
\end{array}
\end{math}
\end{footnotesize}
\caption{Invasion Percolation\label{f:invperc1}}
\end{figure}

\begin{figure}
\begin{center}
% {\epsfxsize=0.2\textwidth\framebox{\epsffile{fig/invperc1.ps}}}
% {\epsfxsize=0.2\textwidth\framebox{\epsffile{fig/invperc2.ps}}}
% {\epsfxsize=0.2\textwidth\framebox{\epsffile{fig/invperc3.ps}}}
% {\epsfxsize=0.2\textwidth\framebox{\epsffile{fig/invperc4.ps}}}
\end{center}
\caption{Fractal Generated by Invasion Percolation\label{f:invperc2}}
\end{figure}

The simulation continues until
some fixed percentage of cells have been filled,
or until some other condition
(such as the presence of trapped regions)
is achieved.
The fractal structure of the filled and unfilled regions
is then examined to determine how much oil could be recovered.
The na\"{\i}ve way to implement this is to repeatedly scan the array;
a more sophisticated, and much faster, sequential technique
is to maintain a priority queue of unfilled cells
which are neighbors of filled cells.
This latter technique is similar to the list-based methods used in some cellular automaton programs,
and is very difficult to parallelize effectively.

The inputs to this module are:
\begin{description}
\item[{\tt{matrix}}:]
	an integer matrix.
\item[{\tt{nrows, ncols}}:]
	the number of rows and columns in the matrix and mask.
\item[{\tt{nfill}}:]
	the number of points to fill.
\end{description}
Its output is:
\begin{description}
\item[{\tt{mask}}:]
	a Boolean matrix filled with {\tt{true}} (showing a filled cell) or {\tt{false}} (showing an unfilled cell).
\end{description}
Filling begins at the central cell of the matrix
(rounding down for even-sized axes).

\subsection{{\tt{life}}:
	Game of Life
	\label{s:toys-life}}

\begin{figure}
\begin{center}
% {\epsfxsize=0.2\textwidth\framebox{\epsffile{fig/life1.ps}}}
% {\epsfxsize=0.2\textwidth\framebox{\epsffile{fig/life2.ps}}}
% {\epsfxsize=0.2\textwidth\framebox{\epsffile{fig/life3.ps}}}
% {\epsfxsize=0.2\textwidth\framebox{\epsffile{fig/life4.ps}}}
\end{center}
\caption{Snapshots from the Game of Life\label{f:life}}
\end{figure}

This module simulates the evolution of Conway's Game of Life,
a two-dimensional cellular automaton (Figure~\ref{f:life}).
Its inputs are:
\begin{description}
\item[{\tt{matrix}}:]
	a Boolean matrix representing the Life world.
\item[{\tt{nrows, ncols}}:]
	the number of rows and columns in the matrix.
\item[{\tt{numgen}}:]
	the number of generations to simulate.
\end{description}
Its output is:
\begin{description}
\item[{\tt{matrix}}:]
	a Boolean matrix representing the world after simulation.
\end{description}
At each time step,
this module must count the number of live ({\tt{true}}) neighbors
of each cell,
using both orthogonal and diagonal connectivity
and circular boundary conditions.
The update rule is simple:
if a cell has 3 live neighbors,
or has 2 live neighbors and is already alive,
it is alive in the next generation.
In any other situation,
the cell becomes, or stays, dead.

\subsection{{\tt{mandel}}: Mandelbrot Set Generation\label{s:toys-mandel}}

This module generates the Mandelbrot Set for a specified region of the complex plane.

{\inputspec}

\begin{description}
\item[{\tt{nrows, ncols}}:]
	the number of rows and columns in the output matrix.
\item[{\tt{x0, y0}}:]
	the real coordinates of the lower-left corner of the region to be generated.
\item[{\tt{dx, dy}}:]
	the extent of the region to be generated.
\end{description}

{\outputspec}

\begin{description}
\item[{\tt{matrix}}:]
	an integer matrix containing the iteration count at each point in the region.
\end{description}

Given initial coordinates $(x_0, y_0)$,
the Mandelbrot Set is generated by iterating the equation
\begin{eqnarray*}
x^{\prime}	& =	& x^2 - y^2 + y_0	\\
y^{\prime}	& =	& 2{x}{y} + x_0
\end{eqnarray*}
until either an iteration limit is reached,
or the values diverge.
The iteration limit used in this module is 150 steps;
divergence occurs when $x^2 + y^2$ becomes 2.0 or greater.
The integer value of each element of the matrix is
the number of iterations done.

If possible,
the values produced should depend only on the size of the matrix and the seed,
\emph{not} on the number of processors or threads used.

\subsection{{\tt{norm}}: Point Location Normalization\label{s:toys-norm}}

This module normalizes point coordinates so that all points lie within the unit square $[0{\ldots}1]{\times}[0{\ldots}1]$.

{\inputspec}

\begin{description}
\item[{\tt{points}}:]
	a vector of point locations.
\end{description}

{\outputspec}

\begin{description}
\item[{\tt{points}}:]
	a vector of normalized point locations.
\end{description}

If $x_{min}$ and $x_{max}$ are the minimum and maximum $x$ coordinate values in the input vector,
then the normalization equation is:
\[	x_{i}^{\prime} = \frac{x_i - x_{min}}{x_{max} - x_{min}}	\]
$y$ coordinates are normalized in the same fashion.

\subsection{{\tt{outer}}: Outer Product\label{s:toys-outer}}

This module turns a vector containing point positions
into a dense, symmetric, diagonally dominant matrix
by calculating the distances between each pair of points.
It also constructs a real vector whose values are
the distance of each point from the origin.

{\inputspec}

\begin{description}
\item[{\tt{points}}:]
	a vector of $(x,y)$ points, where $x$ and $y$ are the point's position.
\end{description}

{\outputspec}

\begin{description}
\item[{\tt{matrix}}:]
	a real matrix, whose values are filled with inter-point distances.
\item[{\tt{vector}}:]
	a real vector, whose values are filled with origin-to-point distances.
\end{description}

Each matrix element $M_{i,j}$ such that $i \neq j$
is given the value $d_{i, j}$,
the Euclidean distance between point $i$ and point $j$.
The diagonal values $M_{i, i}$ are then set to
{\tt{nelts}} times the maximum off-diagonal value
to ensure that the matrix is diagonally dominant.
The value of the vector element $v_i$ is set to
the distance of point $i$ from the origin,
which is given by $\sqrt{{x_i}^2 + {y_i}^2}$.

\subsection{{\tt{product}}:
	Matrix-Vector Product
	\label{s:toys-product}}

Given a matrix $A$,
a vector $V$,
and an assumed solution $X$ to the equation $AX=V$,
this module calculates the actual product $AX={V^{\prime}}$,
and then finds the magnitude of the error.
Inputs are:
\begin{description}
\item[{\tt{matrix}}:]
	the real matrix $A$.
\item[{\tt{actual}}:]
	the real vector $V$.
\item[{\tt{candidate}}:]
	a real vector containing the supposed solution.
\item[{\tt{nelts}}:]
	the number of values in each vector, and the size of the matrix along each axis.
\end{description}
The output of this function is:
\begin{description}
\item[{\tt{e}}:]
	the largest absolute value in the element-wise difference of $V$ and $V^{\prime}$.
\end{description}

\subsection{{\tt{randmat}}:
	Random Number Generation
	\label{s:toys-rng}}

This module fills a matrix with pseudo-random integers.
The inputs to this module are:
\begin{description}
\item[{\tt{nrows, ncols}}:]
	the number of rows and columns in the matrix.
\item[{\tt{s}}:]
	the random number generation seed.
\end{description}
Its output is:
\begin{description}
\item[{\tt{matrix}}:]
	an integer matrix filled with random values.
\end{description}
Note that,
as in all problems,
the output is required to be independent of the number of processors used.
Generating new seed values from the given seed,
and running one copy of the random number generator
on each processor,
is therefore unlikely to qualify as a solution.

\subsection{{\tt{sor}}: Successive Over-Relaxation\label{s:toys-sor}}

This module solves a matrix equation $AX=V$
for a dense, symmetric, diagonally dominant matrix $A$
and an arbitrary vector non-zero $V$ using successive over-relaxation.

{\inputspec}

\begin{description}
\item[{\tt{matrix}}:]
	the square real matrix $A$.
\item[{\tt{target}}:]
	the real vector $V$.
\item[{\tt{tolerance}}:]
	the solution tolerance, e.g., $10^{-6}$.
\end{description}

{\outputspec}

\begin{description}
\item[{\tt{solution}}:]
	a real vector containing the solution $X$.
\end{description}

\subsection{{\tt{thresh}}:
	Histogram Thresholding
	\label{s:toys-thresh}}

This module performs histogram thresholding on an image.
Given an integer image $I$ and a target percentage $p$,
it constructs a binary image $B$
such that $B_{i,j}$ is set
if no more than $p$ percent of the pixels in $I$ are brighter than $I_{i,j}$.
The general idea is that an image's histogram should have 2 peaks,
one centered around the average foreground intensity,
and one centered around the average background intensity.
This program attempts to set
a threshold between the two peaks in the histogram
and select the pixels above the threshold.
This module's inputs are:
\begin{description}
\item[{\tt{matrix}}:]
	the integer matrix to be thresholded.
\item[{\tt{nrows, ncols}}:]
	the number of rows and columns in the matrix and mask.
\item[{\tt{percent}}:]
	the minimum percentage of cells to retain.
\end{description}
Its output is:
\begin{description}
\item[{\tt{mask}}:]
	a Boolean matrix filled with {\tt{true}} (showing a cell that is kept) or {\tt{false}} (showing a cell that is discarded).
\end{description}

\subsection{{\tt{vecdiff}}:
	Vector Difference
	\label{s:toys-vecdiff}}

This module finds the maximum absolute elementwise difference between two vectors of real numbers.
Its inputs are:
\begin{description}
\item[{\tt{left}}:]
	the first vector.
\item[{\tt{right}}:]
	the second vector.
\end{description}
Its output is:
\begin{description}
\item[{\tt{maxdiff}}:]
	the largest absolute difference between any two corresponding vector elements.
\end{description}

\subsection{{\tt{winnow}}:
	Weighted Point Selection
	\label{s:toys-winnow}}

This module converts a matrix of integer values
to a vector of points,
represented as $x$ and $y$ coordinates.
Its inputs are:
\begin{description}
\item[{\tt{matrix}}:]
	an integer matrix, whose values are used as masses.
\item[{\tt{mask}}:]
	a Boolean matrix showing which points are eligible for consideration.
\item[{\tt{nrows, ncols}}:]
	the number of rows and columns in the matrix.
\item[{\tt{nelts}}:]
	the number of points to select.
\end{description}
Its output is:
\begin{description}
\item[{\tt{points}}:]
	a vector of $(x,y)$ points.
\end{description}
Each location where {\tt{mask}} is {\tt{true}}
becomes a candidate point,
with a weight equal to the integer value in {\tt{matrix}} at that location
and $x$ and $y$ coordinates
equal to its row and column indices.
These candidate points are then sorted into increasing order by weight,
and {\tt{nelts}} evenly-spaced points selected to create the result vector.

