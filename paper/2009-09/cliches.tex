\section{Parallel Clich\'{e}s\label{s:cliche}}

\subsection{Operations\label{s:cliche-ops}}

This list details some operations which are supported by many parallel programming systems.
The toys described in Section~\ref{s:toys} provide opportunities for exercising many of these,
and implementors are encouraged to phrase discussion of their work in terms of these operations where appropriate.

\begin{itemize}
\item	elementwise operations on arrays (unary, binary, and scalar promotion)
\item	cumulative operations on arrays (reduction and parallel prefix)
\item	array re-shaping and re-sizing (e.g.\ sub-array extraction)
\item	partial (conditionally masked) versions of all of the above
\item	regular data motion (e.g.\ circular and planar shifting)
\item	irregular data motion (e.g.\ 1-to-1, 1-to-many, and many-to-1 permutation)
\item	scalar and non-scalar value location (e.g.\ finding a scalar or record value in an array)
\item	differential local addressing (i.e.\ subscripting an array with an array)
\item	full or partial replication of shared read-only values
\item	full or partial replication of shared read-mostly values with automatic consistency management
\item	structures with rotating ownership, suitable for migration
\item	producer-consumer structures
\item	partitionable structures
\item	pre-determined run-time re-partitioning (i.e.\ re-distributing an array)
\item	dynamic re-partitioning (e.g.\ for load balancing)
\item	committed mutual exclusion (e.g.\ waiting on a semaphore)
\item	uncommitted mutual exclusion (e.g.\ lock or fail)
\item	barrier synchronization
\item	multiple concurrent barriers used by non-overlapping groups of processes
\item	fetch-and-add, and other fetch-and-operate functions
\item	pre-scheduled receipt of messages of known composition
\item	variable-length message receipt
\item	message selection by type
\item	message selection by source
\item	message selection by contents
\item	guarded message selection (e.g.\ CSP's {\tt{alt}} construct)
\item	broadcast and partial broadcast
\item	split-phase (non-blocking) operations
\item	data marshalling and unmarshalling of ``flat'' structures (e.g.\ arrays of scalars)
\item	data marshalling and unmarshalling of nested or linked structures
\item	heterogeneous parallelism (i.e.\ running different applications concurrently in one machine)
\item	pipelining
\item	distributed linked structures in systems without physically-shared memory
\item	indexing of distributed shared arrays
\item	collective I/O
\item	uniform-size interleaved I/O
\item	heterogeneous-size interleaved I/O
\item	independent I/O operations on a single shared file
\end{itemize}

\subsection{Memory Reference Patterns\label{s:cliche-memref}}

These are inspired by the categorization used in the Munin system \cite{b:munin-overview}.
Again, implementors are encouraged to phrase discussion of their work in these terms where appropriate.

\begin{itemize}
\item	Write-once:
	variables which are assigned a value once, and only read thereafter.
	These can be supported through replication.
\item	Write-many:
	variables which are written and read many times.
	If several processes write to the variable concurrently, they typically write to different portions of it.
\item	Producer-consumer:
	variables which are written to by one or more objects, and read by one or more other objects.
	Entries in a shared queue, or a mailbox used for communication between two processes, are examples.
\item	Private:
	variables which are potentially shared, but actually private.
	The interior points of a mesh in a program which uses geometric decomposition fall into this category,
	while boundary points belong to the previous class.
\item	Migratory:
	variables which are read and written many times in succession by a single process before ownership passes to another process.
	The structures representing particles in an $N$-body simulation are the clearest example of this category.
\item	Result:
	Accumulators which are updated many times by different processes, but thereafter read.
\item	Read-mostly:
	Any variable which is read much more often than it is written to.
	The best score so far in a search problem is a typical instance of this class:
	while any process might update it, in practice processes read its value much more often than they change it.
\item	Synchronization:
	Variables used to force explicit synchronization in a program, such as semaphores and locks.
	These are typically ignored for long periods, and then the subject of intense bursts of access activity.
\item	General read-write:
	Any variable which cannot be put in one of the above categories.
\end{itemize}
